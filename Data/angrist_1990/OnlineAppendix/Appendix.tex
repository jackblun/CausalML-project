%2multibyte Version: 5.50.0.2953 CodePage: 1251

\documentclass[12pt]{article}
%%%%%%%%%%%%%%%%%%%%%%%%%%%%%%%%%%%%%%%%%%%%%%%%%%%%%%%%%%%%%%%%%%%%%%%%%%%%%%%%%%%%%%%%%%%%%%%%%%%%%%%%%%%%%%%%%%%%%%%%%%%%%%%%%%%%%%%%%%%%%%%%%%%%%%%%%%%%%%%%%%%%%%%%%%%%%%%%%%%%%%%%%%%%%%%%%%%%%%%%%%%%%%%%%%%%%%%%%%%%%%%%%%%%%%%%%%%%%%%%%%%%%%%%%%%%
\usepackage{graphicx}
\usepackage{amsmath}
\usepackage{harvard}
\usepackage{relsize}
\usepackage{tabularx}
\usepackage[english]{babel}

\setcounter{MaxMatrixCols}{10}
%TCIDATA{OutputFilter=LATEX.DLL}
%TCIDATA{Version=5.50.0.2953}
%TCIDATA{Codepage=1251}
%TCIDATA{<META NAME="SaveForMode" CONTENT="1">}
%TCIDATA{BibliographyScheme=Manual}
%TCIDATA{LastRevised=Monday, October 25, 2010 10:41:08}
%TCIDATA{<META NAME="GraphicsSave" CONTENT="32">}
%TCIDATA{CSTFile=LaTeX article (bright).cst}

\newtheorem{theorem}{Theorem}
\newtheorem{algorithm}[theorem]{Algorithm}
\newtheorem{axiom}[theorem]{Axiom}
\newtheorem{case}[theorem]{Case}
\newtheorem{claim}[theorem]{Claim}
\newtheorem{conclusion}[theorem]{Conclusion}
\newtheorem{condition}[theorem]{Condition}
\newtheorem{conjecture}[theorem]{Conjecture}
\newtheorem{corollary}[theorem]{Corollary}
\newtheorem{criterion}[theorem]{Criterion}
\newtheorem{definition}[theorem]{Definition}
\newtheorem{example}[theorem]{Example}
\newtheorem{exercise}[theorem]{Exercise}
\newtheorem{lemma}[theorem]{Lemma}
\newtheorem{notation}[theorem]{Notation}
\newtheorem{problem}[theorem]{Problem}
\newtheorem{proposition}[theorem]{Proposition}
\newtheorem{remark}[theorem]{Remark}
\newtheorem{solution}[theorem]{Solution}
\newtheorem{summary}[theorem]{Summary}
\renewcommand \thesubsection{\Roman{subsection}}
\setlength{\topmargin}{-0.2in} \setlength{\oddsidemargin}{0.2in}
\setlength{\textwidth}{6.3in} \setlength{\textheight}{8.6in}
\input{tcilatex}
\begin{document}

\title{Schooling and the Vietnam-Era GI Bill: Evidence from the Draft Lottery%
}
\author{Joshua D. Angrist and Stacey H. Chen}
\maketitle

\section*{Web Appendix}

\subsection{Sample Notes}

We worked with a non-public-use version of the 1-in-6 long-form sample that
includes date of birth. The long-form sample is the basis for the publicly
available Integrated Public Use Microdata Series (IPUMS) files. These files
are simple random samples drawn from the 1-in-6 file, though the 1-in-6 file
is not a simple random sample from the census sampling frame. Rather, the
Census Bureau reduces the sampling rate in more densely populated areas
(Census Bureau 2005). Adjustment for variation in sampling rates is made
here using the weighting variables that are included in the long-form file.
These weights adjust for non-response and for non-random sampling, and are
designed to match external population totals by age, race, sex, and Hispanic
origin. In practice, weighting matters little for our results. We confirmed
that the means from the 1-in-6 file are close to those from the 5 percent
file distributed via IPUMS. The original 2000 long-form sample includes
Puerto Rico and island territories; residents of these areas are omitted
from our study.

\subsection{Schooling Imputation}

Using a matched CPS file with responses to both old (highest grade
completed) and new (categorical) schooling questions, Jaeger (1997)
calculates average and median highest grade completed conditional on
categorical school values. \ He finds that the conditional median gives a
better fit than the mean. \ We therefore use median highest grade completed
for most categorical values. \ A drawback of this scheme, however, is that
the categories in the new CPS schooling variable differ slightly from those
on the 2000 Census long-form. Specifically, the Census allows for an
additional some-college category: ``some college, but less than one year."
Because some veterans appear to have used the GI Bill to start a college
program which they then left, we would like to distinguish this group from
other veterans when imputing years of schooling. \ This may matter for our
draft-lottery estimates of linear-in-schooling human capital earnings
functions. A second drawback of the Jaeger scheme for our purposes is that
it assigns the same value to those who report finishing 12th grade with no
diploma and those who received a diploma.

In view of these concerns, we used Jaeger's finer conditional mean
imputation to assign values to the census categories ``grade 12 no degree"
and ``one or more years of college". Finally, we estimated a fractional year
for the census category ``some college but less than one year", by assuming
that time in college is exponentially distributed with a fixed dropout
hazard each month. \ This hazard rate was estimated from the ratio of those
with at least 13 years completed to those with at least 13 years enrolled in
the 1980 Census (for men aged 26-36), assuming a fixed hazard for 8 months
of school. The exponential parameter was then used to estimate expected
months in school for those ever enrolled in grade 13 college who drop out
after one year. The result is an imputed value of 12.55 years. The resulting
imputation scheme is: no schooling (0); nursery school through 4th grade
(2.5); 5th-6th grade (5.5); 7th-8th grade (7.5); 9th (9); 10th grade (10);
11th grade (11); 12th grade no diploma (11.38); high school graduate (12);
some college less than 1 year (12.55); 1 or more years of college no degree
(13.35); associate degree (14); bachelors degree (16); masters, professional
or doctoral degree (18).

A direct application of Jaeger's formula generates results almost identical
to those reported in the paper. \ Note also that estimates of effects of
military service on discrete schooling variables (e.g., an indicator for
college graduation status) are unaffected by the choice of imputation scheme.

\subsection{Additional Tables}

This section includes three additional tables for nonwhites. Table A1 shows
the effect of Vietnam-era veteran status on education, Table A2 reports the
effect on education by birth year, and Table A3 shows the effect on labor
market variables and other outcomes.

\end{document}
